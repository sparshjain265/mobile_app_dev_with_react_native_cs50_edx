\chapter{JavaScript Overview}
\section{Introduction}
JavaScript is Interpreted!
\begin{itemize}
	\item Each browser has its own JavaScript engine, which either interprets the code,
	      or uses some sort of lazy compilation.
	      \begin{itemize}
		      \item V8: Chrome and Node.js
		      \item SpiderMonkey: Firefox
		      \item JavaScriptCore: Safari
		      \item Chakra: Microsoft Edge/IE
	      \end{itemize}
	\item They each implement the ECMAScript standard, but may differ for anything not
	      defined by the standard.
\end{itemize}

\section{Syntax}
Semicolons are optional!
\begin{code}
	\inputminted{js}{src0/0-syntax.js}
	\caption{JavaScript Syntax}
\end{code}

\section{Types}
\begin{itemize}
	\item Dynamic Typing
	\item Primitive Types (no methods, immutable)
	      \begin{itemize}
		      \item undefined
		      \item null
		      \item boolean
		      \item number
		      \item string
		      \item (symbol) (New, in ES6)
	      \end{itemize}
	\item Objects
	\item Typecasting? Coercion.
	\item Explicit vs. Implicit Coercion
	      \begin{itemize}
		      \item \mintinline{js}{const x = 42;}
		      \item \mintinline{js}{const explicit = String(x); // explicit === '42'}
		      \item \mintinline{js}{const implicit = x + ''; // implicit === '42'}
	      \end{itemize}
	\item == vs. ===
	      \begin{itemize}
		      \item == coerces the types
		      \item === requires equivalent types
	      \end{itemize}
	\item Falsy values? return false when cast to boolean
	      \begin{itemize}
		      \item undefined
		      \item null
		      \item false
		      \item +0, -0, NaN
		      \item ""
	      \end{itemize}
	\item Truthy?
	      \begin{itemize}
		      \item \text{[]}
		      \item \{\}
		      \item literally everything except falsy values
	      \end{itemize}
\end{itemize}

\begin{code}
	\inputminted{js}{src0/1-types.js}
	\caption{JavaScript Types}
\end{code}

% \clearpage
\section{Objects}
\begin{itemize}
	\item Everything (except primitive types) are objects
	\item Prototypal Inheritance
\end{itemize}

\subsection{Primitives vs. Objects}
\begin{itemize}
	\item Primitives are immutable
	\item Objects are mutable and stored by reference
	\item Passing by reference vs. passing by value
\end{itemize}

\begin{code}
	\inputminted{js}{src0/2-objects.js}
	\caption{JavaScript Objects}
\end{code}

\clearpage
\begin{code}
	\inputminted{js}{src0/3-objectMutation.js}
	\caption{Object Mutation}
\end{code}

\section{Prototypal Inheritance}
\begin{itemize}
	\item Non-primitive types have a few properties/methods associated with them.
	      \begin{itemize}
		      \item Array.prototype.push()
		      \item String.prototype.toUpperCase()
	      \end{itemize}
	\item Each object stores a reference to its prototype
	\item Properties/methods defined most tightly to the instance have priority
	\item Most primitive types have object wrappers
	      \begin{itemize}
		      \item String()
		      \item Number()
		      \item Boolean()
		      \item Object()
		      \item (Symbol())
	      \end{itemize}
	\item JS will automatically "box" (wrap) primitive values so you have access
	      to methods
	      \begin{minted}{js}
				42.toString()		// Errors
				const x = 42;
				x.toString()		// "42"
				x.__proto__			// [Number: 0]
				x instanceof Number	// false
			  \end{minted}
	\item Why use reference to a prototype?
	\item What's the alternative?
	\item What's the danger?
\end{itemize}

\section{Scope}
\begin{itemize}
	\item Variable Lifetime
	      \begin{itemize}
		      \item Lexical scoping (var): from when they're declared until when
		            their function ends
		      \item Block scoping (const, let): until the next \} is reached
	      \end{itemize}
	      \begin{code}
		      \inputminted{js}{src0/4-scopeVariables.js}
		      \caption{Variable Scopes}
	      \end{code}
	\item Hoisting (hoisting the definitions to the top)
	      \begin{itemize}
		      \item Function definitions are hoisted, but not lexically-scoped
		            initializations
	      \end{itemize}
	      \begin{code}
		      \inputminted{js}{src0/5-scopeFunctions.js}
		      \caption{Function Scopes}
	      \end{code}
	\item Why? How? JavaScript Engine!
\end{itemize}

\section{JavaScript Engine}
\begin{itemize}
	\item Before executing the code, the engine reads the entire file and
	      will throw a syntax error if one is found.
	      \begin{itemize}
		      \item Any function definitions will be saved in memory
		      \item Variable initializations will not be run, but lexically
		            scoped variable names will be declared
	      \end{itemize}
	\item Execution Phase
\end{itemize}

\section{The Global Object}
\begin{itemize}
	\item All variables and functions are actually parameters and methods
	      on the global object
	      \begin{itemize}
		      \item Browser global object is the `window' object
		      \item Node.js global object is the `global' object
	      \end{itemize}
\end{itemize}

\section{Closures}
\begin{itemize}
	\item Functions that refer to variables declared by parent function
	\item Possible because of scoping
\end{itemize}

\begin{code}
	\begin{minted}{js}
		function makeFunctionArray() {
			const arr = []

			for (var i = 0; i < 5; i++) {
				arr.push(function() { console.log(i) })
			}

			return arr
		}

		const arr = makeFunctionArray()

		arr[0]()
	\end{minted}
	\caption{JavaScript Closure}
\end{code}