\chapter{React, Props, State}
\section{Classes}
\begin{itemize}
	\item Syntax introduced in ES6
	\item Simplifies the defining of complex objects with their own prototypes
	\item Classes vs instances
	\item Methods vs static methods vs properties
	\item new, constructor, extends, super
\end{itemize}

\begin{code}
	\inputminted{js}{src2/1-Set.js}
	\caption{Class Example (Set) in JavaScript}
\end{code}

\begin{code}
	\inputminted{js}{src2/2-Set.js}
	\caption{Extending JS Set Class}
\end{code}

\clearpage
\begin{code}
	\inputminted{js}{src2/3-Todo.js}
	\caption{Using Class for Todo App}
\end{code}

\section{React}
\begin{itemize}
	\item Allows us to write delcarative views that "react" to changes in data
	\item Allows us to abstract complex problems into smaller components
	\item Allows us to write simple code that is still performant
\end{itemize}

\subsection{Imperative vs Declarative}
\begin{itemize}
	\item How vs What
	\item Imperative programming outlines a series of steps to get to what you want
	\item Declarative programming just declares what you want
\end{itemize}

\clearpage
\begin{code}
	\inputminted{js}{src2/4-imperativeGuitar.js}
	\caption{Building Guitar - The Imperative Way}
\end{code}

\clearpage
\begin{code}
	\inputminted{js}{src2/5-declarativeGuitar.js}
	\caption{Building Guitar - The Declarative Way}
\end{code}

\subsection{React is Declarative}
\begin{itemize}
	\item Imperative vs Declarative
	\item The browser APIs aren't fun to work with
	\item React allows us to write what we want, and the library will
	      take care of the DOM manipulation
\end{itemize}

\begin{code}
	\inputminted{js}{src2/6-imperativeSlide.js}
	\caption{Imperative Slide}
\end{code}

\begin{code}
	\inputminted{js}{src2/7-declarativeSlide.js}
	\caption{Declarative Slide}
\end{code}

\subsection{React is Easily Componentized}
\begin{itemize}
	\item Breaking a complex problems into discrete components
	\item Can reuse these components
	      \begin{itemize}
		      \item Consistency
		      \item Iteration speed
	      \end{itemize}
	\item React's declarative nature makes it easy to customize components
\end{itemize}

\begin{code}
	\inputminted{html}{src2/8-slideshow.html}
	\caption{HTML Slideshow}
\end{code}

\begin{code}
	\inputminted{js}{src2/9-slideshowComponents.js}
	\caption{React Slideshow}
\end{code}

\subsection{React is Performant}
\begin{itemize}
	\item We write what we want and React will do the hard work
	\item Reconciliation - the process by which React syncs changes in
	      app state to DOM
	      \begin{itemize}
		      \item Reconstructs the virtual DOM
		      \item Diffs the virtual DOM against the DOM
		      \item Only makes the changes needed*
	      \end{itemize}
\end{itemize}

\section{Writing React}
\begin{itemize}
	\item JSX
	      \begin{itemize}
		      \item XML-like syntax extension of JavaScript
		      \item Transpiles to JavaScript
		      \item Lowercase tags are treated as HTML/SVG tags, uppercase are
		            treated as custom components
	      \end{itemize}
	      \clearpage
	\item Components are just functions
	      \begin{itemize}
		      \item Returns a node (something React can render, e.g. a <div />)
		      \item Receives an object of the properties that are passed to the
		            element
	      \end{itemize}
\end{itemize}
\paragraph{Note:} Can run/try react on \href{https://codesandbox.io}{codesandbox.io} .

\section{Props}
\begin{itemize}
	\item Passed as an object to a component and used to compute the returned node
	\item Changes in these props will cause a recomputation of the returned node
	      ("render")
	\item Unlike in HTML, these can be any JS value (use \{to let react know\})
\end{itemize}

\begin{code}
	\inputminted{js}{src2/a-props.js}
	\caption{Props in React}
\end{code}

\section{State}
\begin{itemize}
	\item Adds internally-managed configuration for a component
	\item `this.state' is a class property on the component instance
	\item Can only be updated by invoking `this.setState()'
	      \begin{itemize}
		      \item Implemented in React.Component
		      \item setState() calls are batched and run asynchronously
		      \item Pass an object to be merged, or a function of previous state
	      \end{itemize}
	\item Changes in state also cause re-renders
\end{itemize}

\begin{code}
	\inputminted{js}{src2/b-state.js}
	\caption{React States}
\end{code}

\clearpage
\section{Todo App}
\begin{enumerate}
	\item Layout what you need
	      \inputminted{js}{src2/todoApp0.js}
	\item Componentize
	      \inputminted{js}{src2/todoApp1.js}
	\item Write Declaratively (Inner HTML)
	      \inputminted{js}{src2/todoApp2.js}
	\item Store todo list in a Data Structure
	      \inputminted{js}{src2/todoApp3.js}
	\item React it
	      \begin{code}
		      \inputminted{js}{src2/todoApp4-react.js}
		      \caption{Todo App in React}
	      \end{code}
\end{enumerate}

\section{React Native}
Why limit React to just web? Bring it to mobile!
\begin{itemize}
	\item A framework that relies on React core
	\item Allows us build mobile apps using only JavaScript
	      \begin{itemize}
		      \item Learn once, write anywhere
	      \end{itemize}
	\item Supports iOS and Android
\end{itemize}